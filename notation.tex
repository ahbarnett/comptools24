% to compile: pdflatex -shell-escape notation && ac notation.pdf &

\documentclass[t]{beamer}
\usepackage[utf8]{inputenc}

\usetheme{CCM}
\usepackage{epstopdf}  % handle EPS
\usepackage{bm}

% alex macros
\newcommand{\ft}[1]{\frametitle{#1}}
\newcommand{\bi}{\begin{itemize}}
\newcommand{\ei}{\end{itemize}}
\newcommand{\ben}{\begin{enumerate}}
\newcommand{\een}{\end{enumerate}}
\newcommand{\be}{\begin{equation}}
\newcommand{\ee}{\end{equation}}
\newcommand{\bea}{\begin{eqnarray}}
\newcommand{\eea}{\end{eqnarray}}
\newcommand{\bc}{\begin{center}}
\newcommand{\ec}{\end{center}}
\newcommand{\mbf}[1]{{\bm #1}}           % requires bm package
\newtheorem{thm}{Theorem}
\newcommand{\ig}[2]{\includegraphics[#1]{#2}}
\newcommand{\tbox}[1]{{\mbox{\tiny #1}}}
\newcommand{\who}[1]{{\scriptsize \textcolor{darkgreen}{(#1)}}}  % cite
\newcommand{\com}[1]{{\scriptsize \textcolor{purple}{#1}}}      % comment
\newcommand{\co}[1]{{\scriptsize \tt #1}}          % code
\newcommand{\vg}{\vspace{2ex}}
\newcommand{\sg}{\vspace{1ex}}
\newcommand{\hg}{\vspace{0.5ex}}
\newcommand{\dr}[1]{{\color{darkred}#1}}    % for bullets
\newcommand{\rb}{\ensuremath{\textcolor{darkred}{\bullet\;}}\ }
\newcommand{\bmp}[1]{\begin{minipage}{#1}}
\newcommand{\bmpt}[1]{\begin{minipage}[t]{#1}}
\newcommand{\emp}{\end{minipage}}
\newcommand{\pig}[2]{\bmp{#1}\includegraphics[width=#1]{#2}\emp} % mp-fig, nogap
\newcommand{\pigm}[3]{\bmp{#1}\href{#3}{\includegraphics[width=#1]{#2}}\emp} % w/ movie
\newcommand{\ora}[1]{{\color{orange} #1}}
\newcommand{\gre}[1]{{\color{green} #1}}
\newcommand{\yel}[1]{{\color{yellow} #1}}
\newcommand{\sr}[1]{{\scriptsize #1}}
\newcommand{\eps}{\epsilon}
\newcommand{\qqquad}{\qquad\qquad}
\newcommand{\qqqquad}{\qqquad\qqquad}
\newcommand{\R}{\mathbb{R}}
\newcommand{\Z}{\mathbb{Z}}
\newcommand{\N}{\mathbb{N}}
\newcommand{\emach}{\epsilon_\tbox{mach}}
\DeclareMathOperator{\im}{Im}
\DeclareMathOperator{\re}{Re}
\newcommand{\bigO}{{\cal O}}
\newcommand{\pO}{{\partial\Omega}}
\newcommand{\xx}{\mbf{x}}
\newcommand{\yy}{\mbf{y}}
\newcommand{\uu}{\mbf{u}}
\newcommand{\nn}{\mbf{n}}
\newcommand{\Srep}{{\cal S}}
\newcommand{\Drep}{{\cal D}}



\title{Notations for Comp.\ Tools 2024 BIE workshop}
\date{June 10, 2024}
\author{\textbf{Alex Barnett}\inst{1}}
\institute{\inst{1} Center for Computational Mathematics, Flatiron Institute, Simons Foundation}

\begin{document}

\begin{frame}
	\titlepage
\end{frame}

\begin{frame}\ft{Introduction}

We aim for notations common in the field of high-order Nystr\"om BIE methods.
  
These follow Kress, and various Stokes BIE literature.

\vg

If you need to break with one notation convention
in your slides, so be it, but
try not to break many conventions at once (it will guaranteed
confuse the audience).

\sg

If you want to change our actual convention in this document, discuss in GitHub Issues, or latex comments here.

\end{frame}
 
\begin{frame}\ft{Geometry and PDE}

  Coordinate $\xx = (x_1,\dots,x_d)$ in $\R^d$. For us $d=2$ or $3$.

  Laplacian $\Delta = \sum_{i=1}^d \partial_{x_i}^2$

  Bounded domain is $\Omega$ (usually this is open, not including boundary)
  
  Boundary curve or surface (interface) is $\pO$ or $\Gamma$.

  \quad\com{Unit outward surface normal $\nn$.  Normal deriv $u_n = \frac{\partial u}{\partial n} = \partial_nu = \nn \cdot \nabla u$.}

\quad\com{We avoid $\nu$ for normal. We avoid $\gamma_0$ or $\gamma_1$ for boundary trace operators since too abstract.}
  
  Exterior domain is $\Omega_e$ or $\R^d \backslash \overline{\Omega}$ or $E$.

  Multiple domains are $\Omega_r$, $r=1,2,\dots$

  Homogeneous PDE is ${\cal L}u = 0$, e.g., ${\cal L} = -\Delta$.

  Scalar-valued solution $u$ preferred (various component potentials can be $v$, $\phi$ etc)

  Vector-valued: $\uu$ (flow velocity or displacement), $\mbf{E}$, $\mbf{H}$ for Maxwell\dots

  Stokes: $(\uu,p)$ is a solution to the Stokes system, $p$ = pressure

  In short: use {\bf bold} for vectors.
  
\end{frame}

\begin{frame}\ft{Potential theory}

  Crucial:  $\xx$ = target, vs $\yy$ = source. Surface quadrature nodes $\yy_j \in \Gamma$.

  \quad\com{will need to teach that $(x_1,x_2)$ is a 2D location, confusion vs $(x_1,y_1)$ common}
  
  Fundamental solutions $\Phi(\xx,\yy)$ or $\Phi(\xx-\yy)$ if translationally invariant.

  Or: $G(\xx,\yy)$.  Use subscript for parameter, eg, $G_\omega$ at wavenumber $\omega$.

  So, ${\cal L}\Phi = \delta$

  Density various greek symbols: $\sigma$, $\tau$, $\mu$ \dots; others in contexts $\mbf{f}$ for Stokes
  
  Scalar single-layer potential
  $(\Srep\sigma)(\xx) := \int_\Gamma \Phi(\xx,\yy) \sigma(\yy) ds_\yy$

\quad  \com{note: $\xx$ is target. surface element in 3D could be $dS_\yy$. The
    subscript $\yy$ is needed to indicate element wrt which coordinate.}

  Scalar double-layer potential
  $(\Drep\sigma)(\xx) := \int_\Gamma
  \frac{\partial \Phi(\xx,\yy)}{\partial n_\yy} \sigma(\yy) ds_\yy$

  \quad \com{note: normal derivative is wrt 2nd argument, if the 2-argument form is used. Needs explaining to newcomers.}

    \quad\com{$\Drep\sigma$, or $\Drep[\sigma]$, is a potential, which can be evaluated anywhere, viz $(\Drep\sigma)(\xx)$.}

Boundary integral operators: $S$ is restriction of $\Srep$ to targets on $\Gamma$.

\quad\com{crucial: Roman $S$ for on-surface {\em operator}, vs Cal script $\Srep\sigma$ is a {\em representation} (potential)}

BIO $S_{\Gamma_i,\Gamma_j}$ has source on $\Gamma_j$, target restricted to $\Gamma_i$

\end{frame}



\begin{frame}\ft{Kernels and jumps}

Kernel of any BIO eg $S$ can be abbreviated $S(\xx,\yy)$ (abusing notation)

Target-derivative operators $S'$ or $D^T$ has kernel $\frac{\partial \Phi(\xx,\yy)}{\partial n_\xx}$

Hypersingular $D'$ or $T$ has kernel $\frac{\partial^2 \Phi(\xx,\yy)}{\partial n_\xx \partial n_\yy}$

\quad\com{Euro Galerkin notation for $S$, $D$, $D^T$, $T$ is respectively: $V$, $K$, $K'$ and $W$}
  
If $\xx\in\Gamma$, $u^{+}(\xx) = \lim_{h\to0^+} u(\xx + h\nn)$ exterior limit

$u^{-}(\xx) = \lim_{h\to0^+} u(\xx - h\nn)$ interior limit. Jump $[u] = u^+ - u^-$.

$u_n^{\pm}(\xx) = \lim_{h\to0^+} \nn_\xx \cdot \nabla u (\xx \pm h\nn)$ normal deriv limits

\quad\com{Example jump relation $[\Drep\sigma] = \sigma$}

$D$ is taken in the principal value sense.

Stokes traction is $\mbf{T}(\uu,p)$, the equivalent of normal derivative.
Abbreviate by $\mbf{T}$.


\end{frame}



\begin{frame}\ft{Miscellaneous}

$N$ for total number of (scalar or vector) unknowns

quadrature weights $w_j$. $j$ for quadrature node index.
  
$A$ for system matrix. $I$ for identity operator or matrix.

$\bm{\sigma} = \{\sigma(\yy_j)\}_{j=1}^N$ for discretized density vector.

\quad\com{This conflicts with vector unknowns eg $\bm{\sigma}$ Stokes
  SLP density. Workarounds welcome}
  
Pieces of boundary (patches) can be $\Gamma_j$ or $\gamma_j$.

Approximate quantities $\tilde{u}$ vs mathematical solution $u$.

$k$ for iterates. $\kappa$ or $\omega$ for wavenumber. $\mu$ for viscosity.

\end{frame}

\end{document}
